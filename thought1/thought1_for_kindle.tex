\documentclass[20pt,]{extarticle}
\usepackage[margin=0.6in]{geometry}
\pagenumbering{gobble}
\usepackage{lmodern}
\usepackage{amssymb,amsmath}
\usepackage{ifxetex,ifluatex}
\usepackage{fixltx2e} % provides \textsubscript
\ifnum 0\ifxetex 1\fi\ifluatex 1\fi=0 % if pdftex
  \usepackage[T1]{fontenc}
  \usepackage[utf8]{inputenc}
\else % if luatex or xelatex
  \ifxetex
    \usepackage{mathspec}
  \else
    \usepackage{fontspec}
  \fi
  \defaultfontfeatures{Ligatures=TeX,Scale=MatchLowercase}
\fi
% use upquote if available, for straight quotes in verbatim environments
\IfFileExists{upquote.sty}{\usepackage{upquote}}{}
% use microtype if available
\IfFileExists{microtype.sty}{%
\usepackage{microtype}
\UseMicrotypeSet[protrusion]{basicmath} % disable protrusion for tt fonts
}{}
\usepackage{hyperref}
\PassOptionsToPackage{usenames,dvipsnames}{color} % color is loaded by hyperref
\hypersetup{unicode=true,
            pdftitle={A Model of Human Thought: Philosophy},
            pdfauthor={Tyler Neylon},
            colorlinks=true,
            linkcolor=black,
            citecolor=Blue,
            urlcolor=Blue,
            breaklinks=true}
\urlstyle{same}  % don't use monospace font for urls
\IfFileExists{parskip.sty}{%
\usepackage{parskip}
}{% else
\setlength{\parindent}{0pt}
\setlength{\parskip}{6pt plus 2pt minus 1pt}
}
\setlength{\emergencystretch}{3em}  % prevent overfull lines
\providecommand{\tightlist}{%
  \setlength{\itemsep}{0pt}\setlength{\parskip}{0pt}}
\setcounter{secnumdepth}{5}
% Redefines (sub)paragraphs to behave more like sections
\ifx\paragraph\undefined\else
\let\oldparagraph\paragraph
\renewcommand{\paragraph}[1]{\oldparagraph{#1}\mbox{}}
\fi
\ifx\subparagraph\undefined\else
\let\oldsubparagraph\subparagraph
\renewcommand{\subparagraph}[1]{\oldsubparagraph{#1}\mbox{}}
\fi
\usepackage{subfig}
\AtBeginDocument{%
\renewcommand*\figurename{Figure}
\renewcommand*\tablename{Table}
}
\AtBeginDocument{%
\renewcommand*\listfigurename{List of Figures}
\renewcommand*\listtablename{List of Tables}
}
\usepackage{float}
\floatstyle{ruled}
\makeatletter
\@ifundefined{c@chapter}{\newfloat{codelisting}{h}{lop}}{\newfloat{codelisting}{h}{lop}[chapter]}
\makeatother
\floatname{codelisting}{Listing}
\newcommand*\listoflistings{\listof{codelisting}{List of Listings}}

\title{A Model of Human Thought: Philosophy}
\author{Tyler Neylon}
\date{216.2018}

% Begin custom, non-pandoc commands.

\newcommand{\latexonlyrule}{\rule}
\newenvironment{densearray}{\begin{array}{rcl}}{\end{array}}
\newcommand{\class}[1]{}
\newcommand{\Rule}[3]{}
\newcommand{\optquad}{\quad}
\newcommand{\smallscrneg}{}
\newcommand{\smallscr}[1]{}
\newcommand{\bigscr}[1]{#1}
\newcommand{\smallscrskip}[1]{}

% End custom, non-pandoc commands.

\begin{document}
\maketitle

\newcommand{\R}{\mathbb{R}}
\newcommand{\Z}{\mathbb{Z}}
\newcommand{\eqnset}[1]{\left.\mbox{$#1$}\;\;\right\rbrace\class{postbrace}{ }}
\providecommand{\optquad}{\class{optquad}{}}
\providecommand{\smallscrneg}{\class{smallscrneg}{ }}
\providecommand{\bigscr}[1]{\class{bigscr}{#1}}
\providecommand{\smallscr}[1]{\class{smallscr}{#1}}
\providecommand{\smallscrskip}[1]{\class{smallscrskip}{\hskip #1}}

What does it mean for something to be true?

What can we do to verify that something is true?

How can we discover new truths?

This article is about a conceptual model of truth that may help answer
these questions. I'm personally motivated to consider these questions
because I'd like to re-evaluate the scientific method --- the search for
truth.

I'll argue that this line of pursuit aligns with an understanding of
human thought itself. As a result, these questions also pertain to
another deep interest of mine, which is writing software that we would
recognize as a person.

\section{A traditional view}\label{a-traditional-view}

The usual view of truth is that an idea is true exactly when it
corresponds to the actual state of the world. A concept you need to
bundle along with any definition of truth is a concept of what kinds of
things are candidates for truth; previous philosophers have argued that
suitable \emph{sentences} or \emph{propositions} are good candidates for
truth. The word \emph{suitable} here refers to the need to avoid tricky
cases such as \emph{``This sentence is false.''} I won't dwell on the
tricky cases because I think they are mostly a distraction from the meat
of the issue.

The reason I'm not a fan of this definition is that it doesn't really
explain much. It doesn't tell us how we can verify truthiness. We're
left asking what it means to correspond to the state of the world. It
seems to simply exchange one word for a few more without helping us
learn anything new in the process.

Rather than arguing that this semi-tautological perspective is
\emph{wrong}, I'll simply provide another that I think is more useful.
Consider it the difference between Roman numerals and our usual Arabic
base-10 decimal notation. It's harder to do long division with Roman
numerals, whereas the structure of a number captured by base-10 notation
is \emph{useful} --- it enables us to take certain actions that might be
more difficult with other notations, even though it adds no new
information. This is the kind of \emph{usefulness} I'm arguing exists in
an alternative model of truth.

\section{A practical view}\label{a-practical-view}

(truth is what is effective; cf relevant / completely true /
social/authority) (degrees of truthiness) (the idea of truth itself as a
useful fiction)

(Charles Pierce / Karl Popper)

\subsubsection{Isn't it true even if no one knows
it?}\label{isnt-it-true-even-if-no-one-knows-it}

What would that even mean? (want a good analogy)

\section{A model of thought}\label{a-model-of-thought}

(goals / questions / answers / actions) (actions as proofs / algorithms)

\section{A bigger picture}\label{a-bigger-picture}

(philosophy as humanity's relationship with the world, not about the
world itself)

(David Hume, Descartes)

(aim to close with the big picture idea and some argument as to why)

\end{document}
