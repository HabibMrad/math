\documentclass[20pt,]{extarticle}
\usepackage[margin=0.6in]{geometry}
\pagenumbering{gobble}
\usepackage{lmodern}
\usepackage{amssymb,amsmath}
\usepackage{ifxetex,ifluatex}
\usepackage{fixltx2e} % provides \textsubscript
\ifnum 0\ifxetex 1\fi\ifluatex 1\fi=0 % if pdftex
  \usepackage[T1]{fontenc}
  \usepackage[utf8]{inputenc}
\else % if luatex or xelatex
  \ifxetex
    \usepackage{mathspec}
  \else
    \usepackage{fontspec}
  \fi
  \defaultfontfeatures{Ligatures=TeX,Scale=MatchLowercase}
\fi
% use upquote if available, for straight quotes in verbatim environments
\IfFileExists{upquote.sty}{\usepackage{upquote}}{}
% use microtype if available
\IfFileExists{microtype.sty}{%
\usepackage{microtype}
\UseMicrotypeSet[protrusion]{basicmath} % disable protrusion for tt fonts
}{}
\usepackage{hyperref}
\PassOptionsToPackage{usenames,dvipsnames}{color} % color is loaded by hyperref
\hypersetup{unicode=true,
            pdftitle={A Model of Human Thought: Philosophy},
            pdfauthor={Tyler Neylon},
            colorlinks=true,
            linkcolor=black,
            citecolor=Blue,
            urlcolor=Blue,
            breaklinks=true}
\urlstyle{same}  % don't use monospace font for urls
\IfFileExists{parskip.sty}{%
\usepackage{parskip}
}{% else
\setlength{\parindent}{0pt}
\setlength{\parskip}{6pt plus 2pt minus 1pt}
}
\setlength{\emergencystretch}{3em}  % prevent overfull lines
\providecommand{\tightlist}{%
  \setlength{\itemsep}{0pt}\setlength{\parskip}{0pt}}
\setcounter{secnumdepth}{5}
% Redefines (sub)paragraphs to behave more like sections
\ifx\paragraph\undefined\else
\let\oldparagraph\paragraph
\renewcommand{\paragraph}[1]{\oldparagraph{#1}\mbox{}}
\fi
\ifx\subparagraph\undefined\else
\let\oldsubparagraph\subparagraph
\renewcommand{\subparagraph}[1]{\oldsubparagraph{#1}\mbox{}}
\fi
\usepackage{subfig}
\AtBeginDocument{%
\renewcommand*\figurename{Figure}
\renewcommand*\tablename{Table}
}
\AtBeginDocument{%
\renewcommand*\listfigurename{List of Figures}
\renewcommand*\listtablename{List of Tables}
}
\usepackage{float}
\floatstyle{ruled}
\makeatletter
\@ifundefined{c@chapter}{\newfloat{codelisting}{h}{lop}}{\newfloat{codelisting}{h}{lop}[chapter]}
\makeatother
\floatname{codelisting}{Listing}
\newcommand*\listoflistings{\listof{codelisting}{List of Listings}}

\title{A Model of Human Thought: Philosophy}
\author{Tyler Neylon}
\date{216.2018}

% Begin custom, non-pandoc commands.

\newcommand{\latexonlyrule}{\rule}
\newenvironment{densearray}{\begin{array}{rcl}}{\end{array}}
\newcommand{\class}[1]{}
\newcommand{\Rule}[3]{}
\newcommand{\optquad}{\quad}
\newcommand{\smallscrneg}{}
\newcommand{\smallscr}[1]{}
\newcommand{\bigscr}[1]{#1}
\newcommand{\smallscrskip}[1]{}

% End custom, non-pandoc commands.

\begin{document}
\maketitle

\newcommand{\R}{\mathbb{R}}
\newcommand{\Z}{\mathbb{Z}}
\newcommand{\eqnset}[1]{\left.\mbox{$#1$}\;\;\right\rbrace\class{postbrace}{ }}
\providecommand{\optquad}{\class{optquad}{}}
\providecommand{\smallscrneg}{\class{smallscrneg}{ }}
\providecommand{\bigscr}[1]{\class{bigscr}{#1}}
\providecommand{\smallscr}[1]{\class{smallscr}{#1}}
\providecommand{\smallscrskip}[1]{\class{smallscrskip}{\hskip #1}}

What does it mean for something to be true?

What can we do to verify that something is true?

How can we discover new truths?

This article is about a conceptual model of truth that may help answer
these questions. I'm personally motivated to consider these questions
because I'd like to re-evaluate the scientific method --- the search for
truth (although thoughts specific to the scientific method will wait for
another article).

I'll argue that this line of thought aligns with an understanding of
human thought itself. As a result, these questions also pertain to
another deep interest of mine, which is writing software that we would
recognize as a person.

\section{A traditional view}\label{a-traditional-view}

The usual view of truth is that an idea is true exactly when it
corresponds to the actual state of the world. Along with a definition of
truth, you need to bring with it a concept of what kinds of things are
candidates for truth; previous philosophers have argued that suitable
\emph{sentences} or \emph{propositions} are good candidates for truth.
The word \emph{suitable} here refers to the need to avoid tricky cases
such as \emph{``This sentence is false.''} I won't dwell on the tricky
cases because I think they are mostly a distraction from the meat of the
issue.

The reason I'm not a fan of this definition is that it doesn't really
explain much. It doesn't tell us how we can verify truthiness. We're
left asking what it means to correspond to the state of the world. It
seems to simply exchange one word for a few more without helping us
learn anything new in the process.

Rather than arguing that this semi-tautological perspective is
\emph{wrong}, I'll simply provide another that I think is more useful.
Consider it the difference between Roman numerals and our usual Arabic
base-10 decimal notation. It's harder to do long division with Roman
numerals, whereas the structure of a number captured by base-10 notation
is \emph{useful} --- it enables us to take certain actions that might be
more difficult with other notations, even though it adds no new
information. This is the kind of \emph{usefulness} I'm arguing exists in
an alternative model of truth.

\section{A practical view}\label{a-practical-view}

I won't be so bold as to redefine truth itself, but I will introduce a
phrase to capture an idea that I think we all implicitly use without
quite naming it. This is the idea of \emph{effective truth}; when I say
something is \emph{effectively true}, I mean that it tends to help us
achieve a goal we have in mind.

An easy example to consider would be to say that a certain book at a
certain store costs \$10. We could achieve our goal of buying that book
by walking into that store with a \$10 bill and paying for that book,
ignoring tax and pretending that people still use paper money.

A more interesting example is to understand the uses and limits of our
own internal model of physics. Suppose you take a cylinder of wood and
roll it on a flat surface. You can reasonably expect it to move
smoothly, gradually slowing down. Imagine doing the same thing with an
empty glass jar --- same result. In fact, most solid cylindrical objects
will act the same, so we have a useful model for predicting how objects
will move on their own based on this.

Now imagine someone gives you a jar half-filled with water, and you also
roll it on a flat surface. If you've never done anything like this
before, then you might be surprised by how uneven the motion is. Even if
you've done this before, you'll find that you have much less certainty
about how it will move than in the other cases. The point is that often
our ideas about how the world work are useful simplifications. In this
case I'd say the simplification is the idea that ``most cylinders act as
if they were solid and had uniform density.''

You might respond that this seems like an arbitrary example that is a
simplification, whereas other ideas are more clearly true or false. I'd
respond by saying that, as far as we know, all of our ideas of parts of
the world are models rather than exact representations. To return for a
moment to the \$10 book, suppose the store burns down before you get
there. Would you say it still costs \$10 at that store? How do you know
they won't raise their prices? What if there was a computer error in
their pricing system, and half of people were charged \$10 and the other
half were charged \$12? These may feel like edge cases, but the point
itself is that unclear edge cases \emph{always} exist.

If you agree that human ideas are imperfect models that help us achieve
our goals, then you can see how it's useful to speak about effective
truths rather than to insist on an idealized version of truth that could
only make perfect sense if it knew all the rules of physics to the
smallest detail.

\subsection{Relevance and social
truths}\label{relevance-and-social-truths}

I want to mention how effective truth relates to two interesting
properties of truth --- \emph{relevance} and \emph{social acceptance}.
These are both interesting aspects of truth that I think make more sense
from the viewpoint of effective truth than they do in the context of
traditional truth.

The \emph{relevance} of an idea seems to be important for that idea to
be discussed as true or false. Suppose a certain particle of dust in a
far-off galaxy may or may not one day collide with another certain
particle of dust. The point of this example is that no one cares.
Neither outcome will have any bearing on any human.

The traditional model of truth applies equally well to these particles
of dust as it does to the more pertinent question of whether or not an
asteroid will collide with Earth and destroy our species. I won't argue
that this is a downfall of the traditional model, but rather that how
humans think is closer to effective truth because we don't bother with
things that don't matter. From a certain point of view we basically
\emph{can't} think about completely irrelevant ideas because we simply
don't see them.

The other property I wanted to mention is \emph{social acceptance,} by
which I mean the sense that some ideas are accepted or not based on
large-scale social behaviors. For example, what kind of clothes are
currently in fashion? While some specific people have more influence on
this than others, this kind of truth is ultimately up to the actions of
the many. Another example of social acceptance are questions of
widely-known fictional characters such as Santa Claus. It's generally
considered true that Santa Claus lives at the North Pole, although there
is no such person. In cases like this, a story has evolved over time.
Perhaps it's been influenced more strongly by some individuals than
others, but the story takes on meaning through its collective acceptance
as a kind of truth.

Similar to relevance, the idea of social acceptance can be viewed as
gaining its value through the lens of effective truth. For example, if
you want to understand the advertising around, or integrate your
children into a world of a Santa Claus-celebrating society, then it's
useful to be aware of the mythology behind Santa Claus. In general,
social truths appear to be ideas that tend to build on top of, or
further enable, connection-building and status-building. Knowing these
truths is effective at achieving social integration.

\subsection{Shades of truthiness}\label{shades-of-truthiness}

From here on, let's work within the view that an idea is true exactly
when it's effective. An interesting result of this view is that truth is
no longer a dichotomy. Instead, just as some ideas are better at helping
us achieve a goal than others, there are similarly many degrees of the
truth of an idea.

One example is the idea that a trip from your apartment in Manhattan to
Penn Station takes 20 minutes. This is a simple idea that, in practice,
may often help you get to Penn Station at a desired moment. But there
are many factors that, when taken into consideration, would make this
idea more accurate. Traffic is worse during rush hour. If the subway is
not running normally, you may need to find an alternative route. The
availability of trains and cars decreases on weekends. If you're aware
of the many moving pieces, then you're in a better position to get to
Penn Station on time. It's not quite \emph{false} that you live 20
minutes away, so much as it is \emph{useful} to act as if that were
true; and in fact even \emph{more useful} to keep in mind modifications
that allow you to plan your trip more resiliantly.

\subsection{Truth itself as a useful
fiction}\label{truth-itself-as-a-useful-fiction}

So if we move away from the dichotomy of truth in favor of thinking in
terms of what's effective, then what becomes of truth itself? Do we
still need it?

I suggest that, in practice, we already use effectiveness as the
building blocks for how we decide to do what we do. Perhaps the idea of
truth itself has come in handy as a counterpoint to lying.

Imagine living in a world without communication. Then the need to
distinguish a truth from a falsehood is far from your thoughts. You
still can learn and accomplish things by experimenting and thinking. In
other words, you already have a model for how the world works without
need for an explicit concept of truth.

Next I'll introduce the idea of a \emph{useful fiction} with an example.
Consider the center of gravity of an object. It is not a part of reality
--- nor do we even pretend it's a physical object. At the same time,
it's useful to think in terms of a center of gravity in order to model
how things behave physically. Many useful concepts can be seen from this
angle --- the idea of ownership, money, responsibility.

What unites these ideas is that they help us get things done without
having an obvious physical counterpoint. We've \emph{added} a new idea
to the world, something that we think in terms of, that is entirely
internal, and that is useful to us. (Again I'll suggest that we are not
even really capable of thinking about things that are entirely useless
to us.)

When it comes to truth, there's no point and counterpoint in reality to
give meaning to the term. The world is only itself. If it has no
falsehood, how could it show us truth?

So we see that the idea of truth must be internal --- something we need
to invent. And in fact, it is less fundamental than simply getting done
what we want to get done.

\subsection{Isn't it true even if no one knows
it?}\label{isnt-it-true-even-if-no-one-knows-it}

At this point I can anticipate a philosophical argument against
effective truth as a more fundamental idea than truth itself. You might
ask about an event that no one was aware of --- is it not true that this
event still happened? In that setting, how is truth an internal idea?

But this line of questioning is misguided. My claim is not about what is
false or true. Rather, the argument of this article is that it's
\emph{more useful} to model truth on effectiveness than on a
correspondence with reality. In other words, I'm never disputing the
truth or falsity of any particular event; I'm not disagreeing with the
traditional concept of truth, rather I'm seeing it as a higher-level
concept that we've invented in a world that already contained the
precepts of thought beforehand.

\subsection{Charles Pierce and Karl
Popper}\label{charles-pierce-and-karl-popper}

I wish I were better versed in the relevant schools of philosophy before
writing this so I could explain exactly how this line of thought fits in
historically. Despite my lack of historical expertise, I will at least
mention a small number of prior ideas and how they relate.

\emph{Pragmatism} is a relevant set of ideas champtioned by Charles
Pierce. Pierce's perspective is typically summarized as saying that an
idea is true exactly when it is \emph{useful}. I have deliberately used
the word \emph{effective} rather than \emph{useful} for reasons I'll
explain below, but the difference is more superficial than profound.

A brief but compelling argument against the idea of truth as usefulness
is this: It may be useful to think of my spouse as faithful, but that
certainly doesn't make it true. I suspect Pierce himself would have
responded to this in much the same way I will, though I can't be certain
of that; so I'll present my reply as my own.

There are two trouble spots with this argument. The first is almost a
distraction, but is important: It's the idea that we can treat belief as
a choice. Do I choose what I think is true? I don't think so at all. We
can choose to act \emph{as if} one or the other thing is true, but even
in acting so, we don't really change our mind unless we receive more
evidence one way or the other (this evidence, admittedly, may be based
on an internal revelation).

The second --- and I think more important --- reply is to keep in mind
what something is useful \emph{for}. If pretending your spouse is
faithful is useful \emph{for} the sake of your peace of mind, then the
argument falls through as belief is not a choice; it would be like
saying that being confident is useful for being confident. On the other
hand, if pretending your spouse is faithful is useful for ensuring that
the only children either of you parent is between the two of you, then
clearly it's \emph{not} useful to simply ``think of'' your spouse as
faithful. Your action of thinking this has essentially no influence over
the child-centered outcome you want to achieve. Your thinking action
fails to ensure your spouse's fidelity in either the traditional or
effective perspectives on truth.

Whatever the case, you need to look at the goals in order to evaluate
the effectiveness of the decisions made to reach those goals. The
correspondence between traditional truth and effective truth is that
both will help you achieve your goals. To be clear, I don't claim that
there is any great difference between pragmatism and effective truth,
but I suspect I'm providing a different point of view by allowing
effective and traditional truth to live in the same world as slightly
different ideas, and to build a model of thought around these. I prefer
the word \emph{effective} over \emph{useful} simply because
\emph{effectiveness} more strongly emphasizes that there is a goal in
mind, and that this goal is brought closer by ideas that are effective.

As a brief note, Karl Popper argued in favor of an idea needing to be
\emph{falsifiable} in order to qualify as a candidate for truth. I agree
that unfalsifiable ideas are not effective ones --- in fact, I would go
even further to say that they are not even \emph{meaningful}. Consider
an unfalsifiable idea. No matter how we act on it, by definition,
nothing in the world will change either way --- otherwise it would be
falsifiable. Because it makes no difference to the world, it can't be an
effective means of achieving any goal. Similarly, if an idea has the
capacity to be effective, then we can try to work toward a goal with it,
and how well we do moving toward that goal provides a kind of
falsifiability. Not to say that these ideas are the same, but rather
that they are compatible in that both Popper and the idea of effective
truth can use falsifiability as a test to throw out particularly
unhelpful ideas.

\section{A model of thought}\label{a-model-of-thought}

The bulk of this article focuses on truth, but these thoughts are even
more interesting in the framework of a model for how humans think. This
is probably a line of thought worthy of its own article, but I wanted to
include an overview here to add some fun context.

The framework is this: Humans have both goals and mental building blocks
of actions. A sequence of these building block actions can help achieve
those goals. Effective truths are these building blocks. We put them
together the way an algorithm built is from subroutines. Put another
way, if we're thinking about a plan of action, then we can see effective
truths as if-then clauses, and a chain of them as a kind of logical
argument (albeit non-boolean since we have shades of truth) which begins
with the state of the world as it is now, and ends with the state of the
world as we'd like it to be.

\emph{Communication} deserves special attention as a particular kind of
building block action. When we communicate or express an idea to another
person, we're trying to do something useful for both of us by working
within a framework where we each have goals. \emph{Statements} are
meaningful in that they are effective by distinguishing among multiple
possible states of the world. In other words, they answer a question,
and their meaning only truly makes sense in the context of both the
\emph{goal} and the \emph{question} they imply.

Although I've been very brief in this section, you can begin to see how
these ideas might become part of a model for both humand and machine
intelligence.

\section{A bigger picture}\label{a-bigger-picture}

Descarte, arguing from first principles, suggested that we can be
certain of nothing beyond our thoughts and our existence. In light of
the non-boolean nature of effective truth, we can see his ideas as
presaging an acceptance of the things we casually consider to be true
as, upon closer examination, mere conveniences we build atop the
unreliable perceptions that we agree to pretend are reality.

Perhaps rebuilding on these ashes left by Descartes, David Hume argued
that philosophy itself is inextricably tied to human nature. Many
philosophical arguments act as if they are fundamentally about the
universe, and that humanity is incidental to them. But, as Hume pointed
out, all of our philosophy exists in human minds, is seen by human eyes,
and is spoken in human terms. It does not exist without us, and the more
we open our eyes to our role, the more we can be honest about how much
we've invented where we'd hoped was discovery.

This link between humanity and philosophy isn't about humans being
special within the world. Rather, we're so ordinary that we must
relinquish hope that our thoughts might have life without us.

\end{document}
