\documentclass[]{article}
\usepackage{lmodern}
\usepackage{amssymb,amsmath}
\usepackage{ifxetex,ifluatex}
\usepackage{fixltx2e} % provides \textsubscript
\ifnum 0\ifxetex 1\fi\ifluatex 1\fi=0 % if pdftex
  \usepackage[T1]{fontenc}
  \usepackage[utf8]{inputenc}
\else % if luatex or xelatex
  \ifxetex
    \usepackage{mathspec}
  \else
    \usepackage{fontspec}
  \fi
  \defaultfontfeatures{Ligatures=TeX,Scale=MatchLowercase}
\fi
% use upquote if available, for straight quotes in verbatim environments
\IfFileExists{upquote.sty}{\usepackage{upquote}}{}
% use microtype if available
\IfFileExists{microtype.sty}{%
\usepackage{microtype}
\UseMicrotypeSet[protrusion]{basicmath} % disable protrusion for tt fonts
}{}
\usepackage{hyperref}
\PassOptionsToPackage{usenames,dvipsnames}{color} % color is loaded by hyperref
\hypersetup{unicode=true,
            pdftitle={Notes on Andrew Ng's CS 229 Machine Learning Course},
            pdfauthor={Tyler Neylon},
            colorlinks=true,
            linkcolor=black,
            citecolor=Blue,
            urlcolor=Blue,
            breaklinks=true}
\urlstyle{same}  % don't use monospace font for urls
\IfFileExists{parskip.sty}{%
\usepackage{parskip}
}{% else
\setlength{\parindent}{0pt}
\setlength{\parskip}{6pt plus 2pt minus 1pt}
}
\setlength{\emergencystretch}{3em}  % prevent overfull lines
\providecommand{\tightlist}{%
  \setlength{\itemsep}{0pt}\setlength{\parskip}{0pt}}
\setcounter{secnumdepth}{5}
% Redefines (sub)paragraphs to behave more like sections
\ifx\paragraph\undefined\else
\let\oldparagraph\paragraph
\renewcommand{\paragraph}[1]{\oldparagraph{#1}\mbox{}}
\fi
\ifx\subparagraph\undefined\else
\let\oldsubparagraph\subparagraph
\renewcommand{\subparagraph}[1]{\oldsubparagraph{#1}\mbox{}}
\fi
\usepackage{subfig}
\AtBeginDocument{%
\renewcommand*\figurename{Figure}
\renewcommand*\tablename{Table}
}
\AtBeginDocument{%
\renewcommand*\listfigurename{List of Figures}
\renewcommand*\listtablename{List of Tables}
}
\usepackage{float}
\floatstyle{ruled}
\makeatletter
\@ifundefined{c@chapter}{\newfloat{codelisting}{h}{lop}}{\newfloat{codelisting}{h}{lop}[chapter]}
\makeatother
\floatname{codelisting}{Listing}
\newcommand*\listoflistings{\listof{codelisting}{List of Listings}}

\title{Notes on Andrew Ng's CS 229 Machine Learning Course}
\author{Tyler Neylon}
\date{331.2016}

% Begin custom, non-pandoc commands.

\newcommand{\latexonlyrule}{\rule}

% End custom, non-pandoc commands.

\begin{document}
\maketitle

\newcommand{\R}{\mathbb{R}}
\newcommand{\eqnset}[1]{\left.\mbox{$#1$}\quad\quad\right\rbrace}
\newcommand{\tr}{\text{tr}\;}

These are notes I'm taking as I review material from Andrew Ng's CS 229
course on machine learning. Specifically, I'm watching
\href{https://www.youtube.com/view_play_list?p=A89DCFA6ADACE599}{these
videos} and looking at the written notes and assignments posted
\href{http://cs229.stanford.edu/materials.html}{here}. These notes are
available in two formats:
\href{http://tylerneylon.com/notes/cs229/cs229.html}{html} and
\href{http://tylerneylon.com/notes/cs229/cs229.pdf}{pdf}.

I'll organize these notes to correspond with the written notes from the
class.

\section{On lecture notes 1}\label{on-lecture-notes-1}

The notes in this section are based on
\href{http://cs229.stanford.edu/notes/cs229-notes1.pdf}{lecture notes
1}.

\subsection{Gradient descent in
general}\label{gradient-descent-in-general}

Given a cost function \(J(\theta)\), the general form of an update is

\[\theta_j := \theta_j - \alpha\frac{\partial J}{\partial \theta_j}.\]

It bothers me that \(\alpha\) is an arbitrary parameter. What is the
best way to choose this parameter? Intuitively, it could be chosen based
on some estimate or actual value of the second derivative of \(J\). What
can be theoretically guaranteed about the rate of convergence under
appropriate conditions?

Why not use Newton's method? A general guess: the second derivative of
\(J\) becomes cumbersome to work with.

It seems worthwhile to keep my eye open for opportunities to apply
improved optimization algorithms in specific cases.

\subsection{Gradient descent on linear
regression}\label{gradient-descent-on-linear-regression}

I realize this is a toy problem because linear regression in practice is
not solve iteratively, but it seems worth understanding well. The
general update equation is, for a single example \(i\),

\[\theta_j := \theta_j + \alpha(y^{(i)} - h_\theta(x^{(i)}))x_j^{(i)}.\]

The delta makes sense in that it is proportional to the error
\(y-h_\theta\), and in that the sign of the product \((y-h_\theta)x\)
guarantees moving in the right direction. However, my first guess would
be that the expression \((y-h_\theta)/x\) would provide a better update.

For example, suppose we have a single data point \((x, y)\) where
\(x\ne 0\), and a random value of \(\theta\). Then a great update would
be

\[\theta_1 := \theta_0 + (y - \theta_0 x)/x,\]

since the next hypothesis value \(h_\theta\) would then be

\[h_\theta = \theta_1 x = \theta_0 x + y - \theta_0x = y,\]

which is good. Another intuitive perspective is that we should be making
\emph{bigger} changes to \(\theta_j\) when \(x_j\) is \emph{small},
since it's harder to influence \(h_\theta\) for such \(x\) values.

This is not yet a solidified intuition. I'd be interested in revisiting
this question if I have time.

\subsection{Properties of the trace
operator}\label{properties-of-the-trace-operator}

The trace of a square matrix obeys the nice property that

\begin{equation}\tr AB = \tr BA.\label{eq:persistent_trace}\end{equation}

One way to see this is to note that

\[\tr AB = a_{ij}b_{ji} = \tr BA,\]

where I'm using the informal shorthand notation that a variable repeated
within a single product implies that the sum is taken over all relevant
values of that variable. Specifically,

\[a_{ij}b_{ji} \;\text{means}\; \sum_{i,j} a_{ij}b_{ji}.\]

I wonder if there's a more elegant way to verify
(\ref{eq:persistent_trace}).

Ng gives other interesting trace-based equations, examined next.

\begin{itemize}
\tightlist
\item
  Goal: \(\quad \nabla_A\tr AB = B^T.\)
\end{itemize}

Since

\[\tr AB = a_{ij}b_{ji},\]

we have that

\[(\nabla_A\tr AB)_{ij} = b_{ji},\]

verifying the goal.

\begin{itemize}
\tightlist
\item
  Goal: \(\quad \nabla_{A^T}f(A) = (\nabla_A f(A))^T.\)
\end{itemize}

This can be viewed as

\[(\nabla_{A^T}f(A))_{ij} = \frac{\partial f}{\partial a_{ji}}
                          = (\nabla_A f(A))_{ji}.\]

\begin{itemize}
\tightlist
\item
  Goal: \(\quad \nabla_A\text{tr}(ABA^TC) = CAB + C^TAB^T.\)
\end{itemize}

I'll use some nonstandard index variable names below because I think it
will help avoid possible confusion. Start with

\[(ABA^TC)_{xy} = a_{xz} b_{zw} a_{vw} c_{vy}.\]

Take the trace of that to arrive at

\[\alpha = \text{tr}(ABA^TC) = a_{xz} b_{zw} a_{vw} c_{vx}.\]

Use the product rule to find \(\frac{\partial\alpha}{\partial a_{ij}}\).
You can think of this as, in the equation above, first setting
\(xz = ij\) for one part of the product rule output, and then setting
\(vw = ij\) for the other part. The result is

\[(\nabla_A\alpha)_{ij} = b_{jw} a_{vw} c_{vi} + a_{xz} b_{zj} c_{ix}
                        = c_{vi} a_{vw} b_{jw} + c_{ix} a_{xz} b_{zj}.\]

(The second equality above is based on the fact that we're free to
rearrange terms within products in the repeated-index notation being
used. Such rearrangement is commutativity of numbers, not of matrices.)

This last expression is exactly the \(ij^\text{th}\) entry of the matrix
\(CAB + C^TAB^T\), which was the goal.

\subsection{Derivation of the normal
equation}\label{derivation-of-the-normal-equation}

Ng starts with

\[\nabla_\theta J(\theta) = \nabla_\theta\frac{1}{2}(X\theta-y)^T(X\theta-y),\]

and uses some trace tricks to get to

\[X^TX\theta - X^Ty.\]

I thought that the trace tricks were not great in the sense that if I
were faced with this problem it would feel like a clever trick out of
thin air to use the trace (perhaps due to my own lack of experience with
matrix derivatives?), and in the sense that the connection between the
definition of \(J(\theta)\) and the result is not clear.

Next is another approach.

Start with
\(\nabla_\theta J(\theta) = \nabla_\theta \frac{1}{2}(\theta^TZ\theta - 2y^TX\theta);\)
where \(Z=X^TX\), and the doubled term is a valid combination of the two
similar terms since they're both real numbers, so we can safely take the
transpose of one of them to add them together.

\newcommand{\nt}{\nabla_\theta}

The left term, \(\theta^T Z\theta\), can be expressed as
\(w=\theta_{i1}Z_{ij}\theta_{j1}\), treating \(\theta\) as an
\((n+1)\times 1\) matrix. Then
\((\nt w)_{i1} = Z_{ij}\theta_{j1} + \theta_{j1}Z_{ji}\), using the
product rule; so \(\nt w = 2Z\theta\), using that \(Z\) is symmetric.

The right term, \(v = y^TX\theta = y_{i1}X_{ij}\theta_{j1}\) with
\((\nt v)_{i1} = y_{j1}X_{ji}\) so that \(\nt v = X^Ty\).

These all lead to \(\nt J(\theta) = X^TX\theta - X^Ty\) as before. I
think it's clearer, though, once you grok the sense that

\[\begin{array}{rcl}
\nt \theta^TZ\theta & = & Z\theta + (\theta^T Z)^T, \text{and} \\
\nt u^T\theta & = & u, \\
\end{array}\]

both as intuitively straightforward matrix versions of derivative
properties.

I suspect this can be made even cleaner since the general product rule

\[\nabla(f\cdot g) = (\nabla f)\cdot g + f \cdot (\nabla g)\]

holds, even in cases where the product \(fg\) is not a scalar; e.g.,
that it is vector- or matrix-valued. But I'm not going to dive into that
right now.

Also note that \(X^TX\) can easily be singular. A simple example is
\(X=0\), the scalar value. However, if \(X\) is \(m\times n\) with rank
\(n\), then \(X^TXe_i = X^Tx^{(i)} \ne 0\) since
\(\langle x^{(i)}, x^{(i)}\rangle \ne 0.\) (If
\(\langle x^{(i)}, x^{(i)}\rangle = 0\) then \(X\) could not have rank
\(n.\)) So \(X^TX\) is nonsingular iff \(X\) has rank \(n\).

Ng says something in a lecture (either 2 or 3) that implied that
\((X^TX)^{-1}X^T\) is \emph{not} the pseudoinverse of \(X\), but for any
real-valued full-rank matrix, it \emph{is}.

\subsection{A probabilistic motivation for least
squares}\label{a-probabilistic-motivation-for-least-squares}

This motivation for least squares makes sense when the error is given by
i.i.d. normal curves, and often this may seem like a natural assumption
based on the central limit theorem.

However, this same line of argument could be used to justify any cost
function of the form

\[J(\theta) = \sum_i f(\theta, x^{(i)}, y^{(i)}),\]

where \(f\) is intuitively some measure of distance between
\(h_\theta(x^{(i)})\) and \(y^{(i)}\). Specifically, model the error
term \(\varepsilon^{(i)}\) as having the probability density function
\(e^{-f(\theta, x^{(i)}, y^{(i)})}\). This is intuitively reasonable
when \(f\) has the aforementioned distance-like property, since error
values are likely near zero and unlikely far from zero. Then the log
likelihood function is

\[\ell(\theta) = \log L(\theta)
               = \log \prod_i e^{-f(\theta, x^{(i)}, y^{(i)})}
               = \sum_i -f(\theta, x^{(i)}, y^{(i)}),\]

so that maximizing \(L(\theta)\) is the same as minimizing the
\(J(\theta)\) defined in terms of this arbitrary function \(f.\) To be
perfectly clear, the motivation Ng provides only works when you have
good reason to believe the error terms are indeed normal. At the same
time, using a nice simple algorithm like least squares is quite
practical even if you don't have a great model for your error terms.

\subsection{Locally weighted linear regression
(LWR)}\label{locally-weighted-linear-regression-lwr}

This idea is that, given a value \(x\), we can choose \(\theta\) to
minimize the modified cost function

\[\sum_i w^{(i)}(y^{(i)}-\theta^Tx^{(i)})^2,\]

where

\[w^{(i)} = \exp\left(-\frac{(x^{(i)}-x)^2}{2\tau^2}\right).\]

I wonder: Why not just compute the value

\[y = \frac{\sum_i w^{(i)}y^{(i)}}{\sum_i w^{(i)}}\]

instead?

I don't have a strong intuition for which approach would be better,
although the linear interpolation is more work (unless I'm missing a
clever way to implement LWR that wouldn't also work for the simpler
equation immediately above). This also reminds me of
\href{https://en.wikipedia.org/wiki/K-nearest_neighbors_algorithm}{the
\(k-\)nearest neighbors algorithm}, though I've seen that presented as a
classification approach while LWR is regression. Nonetheless, perhaps
one could apply a locality-sensitive hash to quickly approximately find
the \(k\) nearest neighbors and then build a regression model using
that.

\hypertarget{refs}{}

\end{document}
